% \chapter{中国科学院大学学位论文撰写要求}
\begin{appendices}\label{sec:appendices}
    \section*{附录代码段合集} \label{sec:testlistings}
    \begin{lstlisting}[
        language=python,
        label=code:catch,
        caption=自动抓取字幕
    ]
    # coding:utf-8
    import sys
    reload(sys)
    sys.setdefaultencoding( "utf-8" )
    
    import scrapy
    from w3lib.html import remove_tags
    from subtitle_crawler.items import SubtitleCrawlerItem
    
    class SubTitleSpider(scrapy.Spider):
        name = "subtitle"
        allowed_domains = ["zimuku.net"]
        start_urls = [
                "http://www.zimuku.net/search?q=&t=onlyst&ad=1&p=20",
                "http://www.zimuku.net/search?q=&t=onlyst&ad=1&p=21",
                "http://www.zimuku.net/search?q=&t=onlyst&ad=1&p=22",
        ]
    
        def parse(self, response):
            hrefs = response.selector.xpath('//div[contains(@class, "persub")]/h1/a/@href').extract()
            for href in hrefs:
                url = response.urljoin(href)
                request = scrapy.Request(url, callback=self.parse_detail)
                yield request
    
        def parse_detail(self, response):
            url = response.selector.xpath('//li[contains(@class, "dlsub")]/div/a/@href').extract()[0]
            print "processing: ", url
            request = scrapy.Request(url, callback=self.parse_file)
            yield request
    
        def parse_file(self, response):
            body = response.body
            item = SubtitleCrawlerItem()
            item['url'] = response.url
            item['body'] = body
            return item
    \end{lstlisting}
    \begin{lstlisting}[
        language=python,
        label=code:catchm,
        caption=主函数
    ]
    class SubtitleCrawlerPipeline(object):
        def process_item(self, item, spider):
            url = item['url']
            file_name = url.replace('/','_').replace(':','_')
            fp = open('result/'+file_name, 'w')
            fp.write(item['body'])
            fp.close()
            return item
    \end{lstlisting}
    \begin{lstlisting}[
        language=python,
        label=code:mv_ass,
        caption=取出扩展名为ass的文件
    ]
    import glob
    import os
    import fnmatch
    import shutil
    import sys
    
    
    def iterfindfiles(path, fnexp):
        for root, dirs, files in os.walk(path):
            for filename in fnmatch.filter(files, fnexp):
                yield os.path.join(root, filename)
    
    
    i=0
    for filename in iterfindfiles(r"./input/", "*.ass"):
        i=i+1
        newfilename = "ass/" + str(i) + "_" + os.path.basename(filename)
        print filename + " <===> " + newfilename
        shutil.move(filename, newfilename)
        #sys.exit(-1)
    
    \end{lstlisting}
    \begin{lstlisting}[
        language=python,
        label=code:mv_Irc,
        caption=取出扩展名为Irc的文件
    ]
    import glob
    import os
    import fnmatch
    import shutil
    import sys
    
    
    def iterfindfiles(path, fnexp):
        for root, dirs, files in os.walk(path):
            for filename in fnmatch.filter(files, fnexp):
                yield os.path.join(root, filename)
    
    
    i=0
    for filename in iterfindfiles(r"./input/", "*.LRC"):
        i=i+1
        newfilename = "lrc/" + str(i) + "_" + os.path.basename(filename)
        print filename + " <===> " + newfilename
        shutil.move(filename, newfilename)
        #sys.exit(-1)
    
    \end{lstlisting}
    \begin{lstlisting}[
        language=python,
        label=code:mv_smi,
        caption=取出扩展名为smi的文件
    ]
    import glob
    import os
    import fnmatch
    import shutil
    import sys
    
    
    def iterfindfiles(path, fnexp):
        for root, dirs, files in os.walk(path):
            for filename in fnmatch.filter(files, fnexp):
                yield os.path.join(root, filename)
    
    
    i=0
    for filename in iterfindfiles(r"./input/", "*.SMI"):
        i=i+1
        newfilename = "smi/" + str(i) + "_" + os.path.basename(filename)
        print filename + " <===> " + newfilename
        shutil.move(filename, newfilename)
        #sys.exit(-1)
    \end{lstlisting}
    \begin{lstlisting}[
        language=python,
        label=code:mv_srt,
        caption=取出扩展名为srt的文件
    ]
    import glob
    import os
    import fnmatch
    import shutil
    import sys
    
    
    def iterfindfiles(path, fnexp):
        for root, dirs, files in os.walk(path):
            for filename in fnmatch.filter(files, fnexp):
                yield os.path.join(root, filename)
    
    
    i=0
    for filename in iterfindfiles(r"./input/", "*.SRT"):
        i=i+1
        newfilename = "srt/" + str(i) + "_" + os.path.basename(filename)
        print filename + " <===> " + newfilename
        shutil.move(filename, newfilename)
        #sys.exit(-1)
    
    \end{lstlisting}
    \begin{lstlisting}[
        language=python,
        label=code:mv_ssa,
        caption=取出扩展名为ssa的文件
    ]
    import glob
    import os
    import fnmatch
    import shutil
    import sys
    
    
    def iterfindfiles(path, fnexp):
        for root, dirs, files in os.walk(path):
            for filename in fnmatch.filter(files, fnexp):
                yield os.path.join(root, filename)
    
    
    i=0
    for filename in iterfindfiles(r"./input/", "*.ssa"):
        i=i+1
        newfilename = "ssa/" + str(i) + "_" + os.path.basename(filename)
        print filename + " <===> " + newfilename
        shutil.move(filename, newfilename)
        #sys.exit(-1)
    
    \end{lstlisting}
    \begin{lstlisting}[
        language=python,
        label=code:mv_str,
        caption=取出扩展名为str的文件
    ]
    import glob
    import os
    import fnmatch
    import shutil
    import sys
    
    
    def iterfindfiles(path, fnexp):
        for root, dirs, files in os.walk(path):
            for filename in fnmatch.filter(files, fnexp):
                yield os.path.join(root, filename)
    
    
    i=0
    for filename in iterfindfiles(r"./input/", "*.str"):
        i=i+1
        newfilename = "str/" + str(i) + "_" + os.path.basename(filename)
        print filename + " <===> " + newfilename
        shutil.move(filename, newfilename)
        #sys.exit(-1)
    
    \end{lstlisting}
    \begin{lstlisting}[
        language=python,
        label=code:mv_sup,
        caption=取出扩展名为sup的文件
    ]
    import glob
    import os
    import fnmatch
    import shutil
    import sys
    
    
    def iterfindfiles(path, fnexp):
        for root, dirs, files in os.walk(path):
            for filename in fnmatch.filter(files, fnexp):
                yield os.path.join(root, filename)
    
    
    i=0
    for filename in iterfindfiles(r"./input/", "*.sup"):
        i=i+1
        newfilename = "sup/" + str(i) + "_" + os.path.basename(filename)
        print filename + " <===> " + newfilename
        shutil.move(filename, newfilename)
        #sys.exit(-1)
    
    \end{lstlisting}
    \begin{lstlisting}[
        language=python,
        label=code:mv_vtt,
        caption=取出扩展名为vtt的文件
    ]
    import glob
    import os
    import fnmatch
    import shutil
    import sys
    
    
    def iterfindfiles(path, fnexp):
        for root, dirs, files in os.walk(path):
            for filename in fnmatch.filter(files, fnexp):
                yield os.path.join(root, filename)
    
    
    i=0
    for filename in iterfindfiles(r"./input/", "*.vtt"):
        i=i+1
        newfilename = "vtt/" + str(i) + "_" + os.path.basename(filename)
        print filename + " <===> " + newfilename
        shutil.move(filename, newfilename)
        #sys.exit(-1)
    \end{lstlisting}
    \begin{lstlisting}[
        language=python,
        label=code:encode,
        caption=编码识别与转码
    ]
    import chardet
    import sys
    import os
    
    if __name__ == '__main__':
        if len(sys.argv) == 2:
            for root, dirs, files in os.walk(sys.argv[1]):
                for file in files:
                    file_path = root + "/" + file
                    f = open(file_path,'r')
                    data = f.read()
                    f.close()
                    encoding = chardet.detect(data)["encoding"]
                    if encoding not in ("UTF-8-SIG", "UTF-16LE", "utf-8", "ascii"):
                        try:
                            gb_content = data.decode("gb18030")
                            gb_content.encode('utf-8')
                            f = open(file_path, 'w')
                            f.write(gb_content.encode('utf-8'))
                            f.close()
                        except:
                            print "except:", file_path
    \end{lstlisting}
    \begin{lstlisting}[
        language=python,
        label=code:extract,
        caption=筛选中文
    ]
    # coding:utf-8
    import chardet
    import os
    import re
    
    cn=ur"([\u4e00-\u9fa5]+)"
    pattern_cn = re.compile(cn)
    jp1=ur"([\u3040-\u309F]+)"
    pattern_jp1 = re.compile(jp1)
    jp2=ur"([\u30A0-\u30FF]+)"
    pattern_jp2 = re.compile(jp2)
    
    for root, dirs, files in os.walk("./srt"):
        file_count = len(files)
        if file_count > 0:
            for index, file in enumerate(files):
                f = open(root + "/" + file, "r")
                content = f.read()
                f.close()
                encoding = chardet.detect(content)["encoding"]
                try:
                    for sentence in content.decode(encoding).split('\n'):
                        if len(sentence) > 0:
                            match_cn =  pattern_cn.findall(sentence)
                            match_jp1 =  pattern_jp1.findall(sentence)
                            match_jp2 =  pattern_jp2.findall(sentence)
                            sentence = sentence.strip()
                            if len(match_cn)>0 and len(match_jp1)==0 and len(match_jp2) == 0 and len(sentence)>1 and len(sentence.split(' ')) < 10:
                                print sentence.encode('utf-8')
                except:
                    continue
    \end{lstlisting}
    \begin{lstlisting}[
        language=python,
        label=code:filter,
        caption=内容过滤
    ]
    # coding:utf-8
    import sys
    import re
    import chardet
    
    if __name__ == '__main__':
        #illegal=r"([\u2000-\u2010]+)"
        illegal=r"([\u0000-\u2010]+)"
        pattern_illegals = [re.compile(r"([\u2000-\u2010]+)"), re.compile(r"([\u0090-\u0099]+)")]
        filters = ["字幕", "时间轴:", "校对:", "翻译:", "后期:", "监制:"]
        filters.append("时间轴:")
        filters.append("校对:")
        filters.append("翻译:")
        filters.append("后期:")
        filters.append("监制:")
        filters.append("禁止用作任何商业盈利行为")
        filters.append("http")
        htmltagregex = re.compile(r'<[^>]+>',re.S)
        brace_regex = re.compile(r'\{.*\}',re.S)
        slash_regex = re.compile(r'\\\w',re.S)
        repeat_regex = re.compile(r'[-=]{10}',re.S)
        f = open("./corpus/all.out", "r")
        count=0
        while True:
            line = f.readline()
            if line:
                line = line.strip()
    
                # 编码识别,不是utf-8就过滤
                gb_content = ''
                try:
                    gb_content = line.decode("utf-8")
                except Exception as e:
                    sys.stderr.write("decode error:  ", line)
                    continue
    
                # 中文识别,不是中文就过滤
                need_continue = False
                for pattern_illegal in pattern_illegals:
                    match_illegal = pattern_illegal.findall(gb_content)
                    if len(match_illegal) > 0:
                        sys.stderr.write("match_illegal error: %s\n" % line)
                        need_continue = True
                        break
                if need_continue:
                    continue
    
                # 关键词过滤
                need_continue = False
                for filter in filters:
                    try:
                        line.index(filter)
                        sys.stderr.write("filter keyword of %s %s\n" % (filter, line))
                        need_continue = True
                        break
                    except:
                        pass
                if need_continue:
                    continue
    
                # 去掉剧集信息
                if re.match('.*第.*季.*', line):
                    sys.stderr.write("filter copora %s\n" % line)
                    continue
                if re.match('.*第.*集.*', line):
                    sys.stderr.write("filter copora %s\n" % line)
                    continue
                if re.match('.*第.*帧.*', line):
                    sys.stderr.write("filter copora %s\n" % line)
                    continue
    
                # 去html标签
                line = htmltagregex.sub('',line)
    
                # 去花括号修饰
                line = brace_regex.sub('', line)
    
                # 去转义
                line = slash_regex.sub('', line)
    
                # 去重复
                new_line = repeat_regex.sub('', line)
                if len(new_line) != len(line):
                    continue
    
                # 去特殊字符
                line = line.replace('-', '').strip()
    
                if len(line) > 0:
                    sys.stdout.write("%s\n" % line)
                count+=1
            else:
                break
        f.close()
        pass
    
    \end{lstlisting}
    \begin{lstlisting}[
        language=java,
        label=code:inject,
        caption=页面注入
    ]
    public void injectQqSettingLayout() {
        XposedHelpers.findAndHookConstructor(QQSettingMe, classLoader,
                BaseActivity,
                QQAppInterface,
                FrameHelperActivity, new XC_MethodHook() {
                    @Override
                    protected void afterHookedMethod(MethodHookParam param) throws Throwable {
                        Field viewsField = XposedUtil.getField(param.thisObject, "a", View[].class);
                        if (viewsField != null) {
                            View[] views = (View []) viewsField.get(param.thisObject);
                            final LinearLayout linearLayout = (LinearLayout) views[0].getParent();
                            LayoutInflater layoutInflater = LayoutInflater.from(mContext);
                            LinearLayout injectLayout = (LinearLayout) layoutInflater.inflate(Setting_Layout_Id, null);
                            ((TextView)injectLayout.findViewById(Setting_TextView_Id)).setText("QQ机器人");
                            injectLayout.setOnClickListener(new View.OnClickListener() {
                                @Override
                                public void onClick(View v) {
                                    Intent intent = new Intent(Intent.ACTION_MAIN);
                                    intent.addCategory(Intent.CATEGORY_LAUNCHER);
                                    ComponentName cn = new ComponentName("cn.EvilCalf.ECBot", "cn.EvilCalf.ECBot.MainActivity");
                                    intent.setComponent(cn);
                                    linearLayout.getContext().startActivity(intent);
                                }
                            });
                            linearLayout.addView(injectLayout);
                        }
                    }
                });
    }
    \end{lstlisting}
    \begin{lstlisting}[
        language=java,
        label=code:socket,
        caption=SOCKET
    ]
    package cn.EvilCalf.ECBot;
    import java.io.IOException;
    import java.net.DatagramPacket;
    import java.net.DatagramSocket;
    import java.net.InetAddress;
    import java.net.SocketException;
    import java.net.UnknownHostException;
    
    public class socket
    {
        protected socket()
        {}
        public static void connectServerWithUDPSocket(String str) {
    
            DatagramSocket socket;
            try {
                socket = new DatagramSocket(1985);
                InetAddress serverAddress = InetAddress.getByName("120.78.197.104");
                byte data[] = str.getBytes();
                DatagramPacket packet = new DatagramPacket(data, data.length ,serverAddress ,10025);
                socket.send(packet);
            } catch (SocketException e) {
                e.printStackTrace();
            } catch (UnknownHostException e) {
                e.printStackTrace();
            } catch (IOException e) {
                e.printStackTrace();
            }
        }
        public static String ReceiveServerSocketData() {
            DatagramSocket socket;
            String result = null;
            try {
                socket = new DatagramSocket(1985);
                byte data[] = new byte[4 * 1024];
                DatagramPacket packet = new DatagramPacket(data, data.length);
                socket.receive(packet);
                result = new String(packet.getData(), packet.getOffset(),
                        packet.getLength());
                socket.close();
            } catch (SocketException e) {
                e.printStackTrace();
            } catch (IOException e) {
                e.printStackTrace();
            }
            return result;
        }
    }
    
    \end{lstlisting}
    \begin{lstlisting}[
        language=java,
        label=code:commnucation,
        caption=消息交互
    ]
    package cn.EvilCalf.ECBot.xposed;

    import android.content.Context;
    
    import com.zhy.http.okhttp.callback.StringCallback;
    
    import java.util.ArrayList;
    import java.util.Map;
    
    import cn.EvilCalf.ECBot.robot.Robot;
    import cn.EvilCalf.ECBot.xposed.utils.XLogUtil;
    import cn.EvilCalf.ECBot.xposed.utils.XPreferenceUtil;
    import cn.EvilCalf.ECBot.xposed.utils.XposedUtil;
    import de.robv.android.xposed.XC_MethodHook;
    import de.robv.android.xposed.XposedHelpers;
    import okhttp3.Call;
    
    import static cn.EvilCalf.ECBot.Common.BaseApplicationImpl;
    import static cn.EvilCalf.ECBot.Common.ChatActivityFacade;
    import static cn.EvilCalf.ECBot.Common.MessageHandlerUtils;
    import static cn.EvilCalf.ECBot.Common.MessageRecord;
    import static cn.EvilCalf.ECBot.Common.QQAppInterface;
    import static cn.EvilCalf.ECBot.Common.SendMsgParams;
    import static cn.EvilCalf.ECBot.Common.SessionInfo;
    import static cn.EvilCalf.ECBot.xposed.utils.XPreferenceUtil.getApiKey;
    import static cn.EvilCalf.ECBot.xposed.utils.XPreferenceUtil.getKeyReply;
    import static de.robv.android.xposed.XposedHelpers.callMethod;
    import static de.robv.android.xposed.XposedHelpers.callStaticMethod;
    import static de.robv.android.xposed.XposedHelpers.findAndHookMethod;
    import static de.robv.android.xposed.XposedHelpers.findClass;
    import static de.robv.android.xposed.XposedHelpers.getObjectField;
    import static de.robv.android.xposed.XposedHelpers.newInstance;
    
    
    
    public class MessageUtil extends BaseHook {
    
        private Object sApplication;
        private Context mContext;
    
        private MessageUtil(ClassLoader classLoader) {
            super(classLoader);
    
        }
    
        MessageUtil(ClassLoader classLoader, Context context) {
            super(classLoader);
            mContext = context;
        }
    
        public void initAutoReply() {
            sApplication = callStaticMethod(findClass(BaseApplicationImpl, classLoader), "getApplication");
            findAndHookMethod(MessageHandlerUtils, classLoader, "a",
                    QQAppInterface,
                    MessageRecord,
                    boolean.class, new XC_MethodHook() {
                        @Override
                        protected void afterHookedMethod(MethodHookParam param) throws Throwable {
                            XLogUtil.d("总开关", XPreferenceUtil.getMasterSwitch() + "");
                            if (!XPreferenceUtil.getMasterSwitch()) return;
    
                            final String frienduin = getObjectField(param.args[1], "frienduin").toString();
                            final String selfuin = getObjectField(param.args[1], "selfuin").toString();
                            final int istroop = (int) getObjectField(param.args[1], "istroop");
                            final String senderuin = getObjectField(param.args[1], "senderuin").toString();
                            final String msg = getObjectField(param.args[1], "msg").toString();
                            boolean isread = (boolean) getObjectField(param.args[1], "isread");
    
                            XLogUtil.d("frienduin", frienduin);
                            XLogUtil.d("selfuin", selfuin);
                            XLogUtil.d("senderuin", senderuin);
    
                            String[] whiteList = XPreferenceUtil.getWhiteList();
    
                            if (whiteList.length != 0 && !senderuin.equals(selfuin)) {
                                if (XPreferenceUtil.getNoReplyTroop() && istroop == 1) return;
                                iteratorWhiteList(frienduin, selfuin, msg, senderuin, istroop, isread, whiteList);
                            }
                        }
                    });
        }
    
        private void iteratorWhiteList(String frienduin, String selfuin, String msg, String senderuin, int istroop, boolean isread, String[] list) {
            for (String s : list) {
                if (frienduin.equals(s) && !isread && !frienduin.equals(selfuin)) {
                    XLogUtil.d("MessageUtil", "iteratorWhiteList is running...");
                    ArrayList<Map<String, String>> keyReply = getKeyReply();
                    if (keyReply != null && keyReply.size() != 0) {
                        for (Map<String, String> stringMap: keyReply){
                            if (stringMap.get("content").trim().equals(msg)) {
                                send(frienduin, selfuin, istroop, stringMap.get("reply_content"));
                                return;
                            }
                        }
                    }
                    Robot robot = new Robot(Robot.RobotType.TULING);
                    robot.setCallback(new ReplyCallback(frienduin, selfuin, istroop));
                    robot.init(getApiKey(), msg, senderuin);
                }
            }
        }
    
        private void send(String frienduin, String selfuin, int istroop, String s) {
            Object qqAppInterface = callMethod(sApplication, "getAppRuntime", selfuin);
            Object sessionInfo = newInstance(findClass(SessionInfo, classLoader));
            Object sendMsgParams = newInstance(findClass(SendMsgParams, classLoader));
            XposedUtil.setField(sessionInfo, "a", frienduin, String.class);
            XposedUtil.setField(sessionInfo, "a", istroop, int.class);
            callStaticMethod(XposedHelpers.findClass(ChatActivityFacade, classLoader), "a", qqAppInterface, mContext, sessionInfo, s, new ArrayList<>(), sendMsgParams);
        }
    
        private class ReplyCallback extends StringCallback {
    
            private String frienduin;
            private String selfuin;
            private int istroop;
    
            ReplyCallback(String frienduin, String selfuin, int istroop) {
                this.frienduin = frienduin;
                this.selfuin = selfuin;
                this.istroop = istroop;
            }
            @Override
            public void onError(Call call, Exception e, int id) {
    
            }
    
            @Override
            public void onResponse(String response, int id) {
                send(frienduin, selfuin, istroop, Robot.tulingReply(response));
            }
        }
    }
    
    \end{lstlisting}
    \begin{lstlisting}[
        language=python,
        label=code:word2rec,
        caption=分词并转化为词向量
    ]
    # coding:utf-8
import sys
import jieba


class WordToken(object):
    def __init__(self):
        self.START_ID = 4
        self.word2id_dict = {}
        self.id2word_dict = {}

    def load_file_list(self, file_list):
        global index
        words_count = {}
        for file in file_list:
            with open(file, 'r', encoding='UTF-8') as file_object:
                for line in file_object.readlines():
                    line = line.strip()
                    seg_list = jieba.cut(line)
                    for str in seg_list:
                        if str in words_count:
                            words_count[str] = words_count[str] + 1
                        else:
                            words_count[str] = 1

        sorted_list = [[v[1], v[0]] for v in words_count.items()]
        sorted_list.sort(reverse=True)
        for index, item in enumerate(sorted_list):
            word = item[1]
            self.word2id_dict[word] = self.START_ID + index
            self.id2word_dict[self.START_ID + index] = word
        return index

    def word2id(self, word):
        if not isinstance(word, str):
            print("Exception: error word not unicode")
            sys.exit(1)
        if word in self.word2id_dict:
            return self.word2id_dict[word]
        else:
            return None

    def id2word(self, i_d):
        i_d = int(i_d)
        if i_d in self.id2word_dict:
            return self.id2word_dict[i_d]
        else:
            return None

    \end{lstlisting}
    \begin{lstlisting}[
        language=java,
        label=code:inject,
        caption=页面注入
    ]
    public void injectQqSettingLayout() {
        XposedHelpers.findAndHookConstructor(QQSettingMe, classLoader,
                BaseActivity,
                QQAppInterface,
                FrameHelperActivity, new XC_MethodHook() {
                    @Override
                    protected void afterHookedMethod(MethodHookParam param) throws Throwable {
                        Field viewsField = XposedUtil.getField(param.thisObject, "a", View[].class);
                        if (viewsField != null) {
                            View[] views = (View []) viewsField.get(param.thisObject);
                            final LinearLayout linearLayout = (LinearLayout) views[0].getParent();
                            LayoutInflater layoutInflater = LayoutInflater.from(mContext);
                            LinearLayout injectLayout = (LinearLayout) layoutInflater.inflate(Setting_Layout_Id, null);
                            ((TextView)injectLayout.findViewById(Setting_TextView_Id)).setText("QQ机器人");
                            injectLayout.setOnClickListener(new View.OnClickListener() {
                                @Override
                                public void onClick(View v) {
                                    Intent intent = new Intent(Intent.ACTION_MAIN);
                                    intent.addCategory(Intent.CATEGORY_LAUNCHER);
                                    ComponentName cn = new ComponentName("cn.EvilCalf.ECBot", "cn.EvilCalf.ECBot.MainActivity");
                                    intent.setComponent(cn);
                                    linearLayout.getContext().startActivity(intent);
                                }
                            });
                            linearLayout.addView(injectLayout);
                        }
                    }
                });
    }
    \end{lstlisting}
    \begin{lstlisting}[
        language=java,
        label=code:get,
        caption=获取域
    ]
    public static Object getObjectField(Object obj, String fieldName) {
		try {
			return findField(obj.getClass(), fieldName).get(obj);
		} catch (IllegalAccessException e) {
			// should not happen
			XposedBridge.log(e);
			throw new IllegalAccessError(e.getMessage());
		} catch (IllegalArgumentException e) {
			throw e;
		}
    }
    public static Field findField(Class<?> clazz, String fieldName) {
		String fullFieldName = clazz.getName() + '#' + fieldName;

		if (fieldCache.containsKey(fullFieldName)) {
			Field field = fieldCache.get(fullFieldName);
			if (field == null)
				throw new NoSuchFieldError(fullFieldName);
			return field;
		}

		try {
			Field field = findFieldRecursiveImpl(clazz, fieldName);
			field.setAccessible(true);
			fieldCache.put(fullFieldName, field);
			return field;
		} catch (NoSuchFieldException e) {
			fieldCache.put(fullFieldName, null);
			throw new NoSuchFieldError(fullFieldName);
		}
	}
    public static Field getField(Object object, String fieldName, Type type) {
        Field[] fields = object.getClass().getDeclaredFields();
        for (Field field : fields) {
            if (field.getType() == type && field.getName().equals(fieldName)) {
                field.setAccessible(true);
                return field;
            }
        }
        return null;
    }
    \end{lstlisting}
\begin{lstlisting}[
    language=java,
    label=code:train,
    caption=训练模型
]
# coding:utf-8
import numpy as np
import tensorflow as tf
from tensorflow.contrib.legacy_seq2seq.python.ops import seq2seq
import word_token
import jieba
import win_unicode_console

win_unicode_console.enable()

input_seq_len = 5
output_seq_len = 5
PAD_ID = 0
GO_ID = 1
EOS_ID = 2
size = 8
init_learning_rate = 1
wordToken = word_token.WordToken()
max_token_id = wordToken.load_file_list(['question', 'answer'])
num_encoder_symbols = max_token_id + 5
num_decoder_symbols = max_token_id + 5


def get_id_list_from(sentence):
    sentence_id_list = []
    seg_list = jieba.cut(sentence)
    for str in seg_list:
        id = wordToken.word2id(str)
        if id:
            sentence_id_list.append(wordToken.word2id(str))
    return sentence_id_list


def get_train_set():
    global num_encoder_symbols, num_decoder_symbols
    train_set = []
    with open('question', 'r', encoding='UTF-8') as question_file:
        with open('answer', 'r', encoding='UTF-8') as answer_file:
            while True:
                question = question_file.readline()
                answer = answer_file.readline()
                if question and answer:
                    question = question.strip()
                    answer = answer.strip()

                    question_id_list = get_id_list_from(question)
                    answer_id_list = get_id_list_from(answer)
                    answer_id_list.append(EOS_ID)
                    train_set.append([question_id_list, answer_id_list])
                else:
                    break
    return train_set


def get_samples(train_set):
    raw_encoder_input = []
    raw_decoder_input = []
    for sample in train_set:
        raw_encoder_input.append([PAD_ID] * (input_seq_len - len(sample[0])) + sample[0])
        raw_decoder_input.append([GO_ID] + sample[1] + [PAD_ID] * (output_seq_len - len(sample[1]) - 1))

    encoder_inputs = []
    decoder_inputs = []
    target_weights = []

    for length_idx in range(input_seq_len):
        encoder_inputs.append(
            np.array([encoder_input[length_idx] for encoder_input in raw_encoder_input], dtype=np.int32))
    for length_idx in range(output_seq_len):
        decoder_inputs.append(
            np.array([decoder_input[length_idx] for decoder_input in raw_decoder_input], dtype=np.int32))
        target_weights.append(np.array([
            0.0 if length_idx == output_seq_len - 1 or decoder_input[length_idx] == PAD_ID else 1.0 for decoder_input in
            raw_decoder_input
        ], dtype=np.float32))
    return encoder_inputs, decoder_inputs, target_weights


def get_model(feed_previous=False):
    learning_rate = tf.Variable(float(init_learning_rate), trainable=False, dtype=tf.float32)
    learning_rate_decay_op = learning_rate.assign(learning_rate * 0.9)

    encoder_inputs = []
    decoder_inputs = []
    target_weights = []
    for i in range(input_seq_len):
        encoder_inputs.append(tf.placeholder(tf.int32, shape=[None], name="encoder{0}".format(i)))
    for i in range(output_seq_len + 1):
        decoder_inputs.append(tf.placeholder(tf.int32, shape=[None], name="decoder{0}".format(i)))
    for i in range(output_seq_len):
        target_weights.append(tf.placeholder(tf.float32, shape=[None], name="weight{0}".format(i)))

    targets = [decoder_inputs[i + 1] for i in range(output_seq_len)]

    cell = tf.contrib.rnn.BasicLSTMCell(size)

    outputs, _ = seq2seq.embedding_attention_seq2seq(
        encoder_inputs,
        decoder_inputs[:output_seq_len],
        cell,
        num_encoder_symbols=num_encoder_symbols,
        num_decoder_symbols=num_decoder_symbols,
        embedding_size=size,
        output_projection=None,
        feed_previous=feed_previous,
        dtype=tf.float32)

    loss = seq2seq.sequence_loss(outputs, targets, target_weights)
    opt = tf.train.GradientDescentOptimizer(learning_rate)
    update = opt.apply_gradients(opt.compute_gradients(loss))
    saver = tf.train.Saver(tf.global_variables())

    return encoder_inputs, decoder_inputs, target_weights, outputs, loss, update, saver, learning_rate_decay_op, learning_rate


def train():
    train_set = get_train_set()
    with tf.Session() as sess:

        sample_encoder_inputs, sample_decoder_inputs, sample_target_weights = get_samples(train_set)
        encoder_inputs, decoder_inputs, target_weights, outputs, loss, update, saver, learning_rate_decay_op, learning_rate = get_model()

        input_feed = {}
        for l in range(input_seq_len):
            input_feed[encoder_inputs[l].name] = sample_encoder_inputs[l]
        for l in range(output_seq_len):
            input_feed[decoder_inputs[l].name] = sample_decoder_inputs[l]
            input_feed[target_weights[l].name] = sample_target_weights[l]
        input_feed[decoder_inputs[output_seq_len].name] = np.zeros([len(sample_decoder_inputs[0])], dtype=np.int32)

        sess.run(tf.global_variables_initializer())

        previous_losses = []
        for step in range(20700):
            [loss_ret, _] = sess.run([loss, update], input_feed)
            if step % 10 == 0:
                print('step=', step, 'loss=', loss_ret, 'learning_rate=', learning_rate.eval())

                if len(previous_losses) > 5 and loss_ret > max(previous_losses[-5:]):
                    sess.run(learning_rate_decay_op)
                previous_losses.append(loss_ret)

                saver.save(sess, './model/EvilCalf')


if __name__ == "__main__":
    train()

\end{lstlisting}
\begin{lstlisting}[
    language=java,
    label=code:predict,
    caption=预测模型
]
# coding:utf-8
import sys
import numpy as np
import tensorflow as tf
from tensorflow.contrib.legacy_seq2seq.python.ops import seq2seq
import word_token
import jieba
import win_unicode_console

win_unicode_console.enable()

input_seq_len = 5
output_seq_len = 5
PAD_ID = 0
GO_ID = 1
EOS_ID = 2
size = 8
init_learning_rate = 1
wordToken = word_token.WordToken()
max_token_id = wordToken.load_file_list(['question', 'answer'])
num_encoder_symbols = max_token_id + 5
num_decoder_symbols = max_token_id + 5


def get_id_list_from(sentence):
    sentence_id_list = []
    seg_list = jieba.cut(sentence)
    for str in seg_list:
        id = wordToken.word2id(str)
        if id:
            sentence_id_list.append(wordToken.word2id(str))
    return sentence_id_list


def seq_to_encoder(input_seq):
    input_seq_array = [int(v) for v in input_seq.split()]
    encoder_input = [PAD_ID] * (input_seq_len - len(input_seq_array)) + input_seq_array
    decoder_input = [GO_ID] + [PAD_ID] * (output_seq_len - 1)
    encoder_inputs = [np.array([v], dtype=np.int32) for v in encoder_input]
    decoder_inputs = [np.array([v], dtype=np.int32) for v in decoder_input]
    target_weights = [np.array([1.0], dtype=np.float32)] * output_seq_len
    return encoder_inputs, decoder_inputs, target_weights


def get_model(feed_previous=False):
    learning_rate = tf.Variable(float(init_learning_rate), trainable=False, dtype=tf.float32)
    learning_rate_decay_op = learning_rate.assign(learning_rate * 0.9)

    encoder_inputs = []
    decoder_inputs = []
    target_weights = []
    for i in range(input_seq_len):
        encoder_inputs.append(tf.placeholder(tf.int32, shape=[None], name="encoder{0}".format(i)))
    for i in range(output_seq_len + 1):
        decoder_inputs.append(tf.placeholder(tf.int32, shape=[None], name="decoder{0}".format(i)))
    for i in range(output_seq_len):
        target_weights.append(tf.placeholder(tf.float32, shape=[None], name="weight{0}".format(i)))

    targets = [decoder_inputs[i + 1] for i in range(output_seq_len)]

    cell = tf.contrib.rnn.BasicLSTMCell(size)

    outputs, _ = seq2seq.embedding_attention_seq2seq(
        encoder_inputs,
        decoder_inputs[:output_seq_len],
        cell,
        num_encoder_symbols=num_encoder_symbols,
        num_decoder_symbols=num_decoder_symbols,
        embedding_size=size,
        output_projection=None,
        feed_previous=feed_previous,
        dtype=tf.float32)

    loss = seq2seq.sequence_loss(outputs, targets, target_weights)
    opt = tf.train.GradientDescentOptimizer(learning_rate)
    update = opt.apply_gradients(opt.compute_gradients(loss))
    saver = tf.train.Saver(tf.global_variables())

    return encoder_inputs, decoder_inputs, target_weights, outputs, loss, update, saver, learning_rate_decay_op, learning_rate


def predict():
    with tf.Session() as sess:
        encoder_inputs, decoder_inputs, target_weights, outputs, loss, update, saver, learning_rate_decay_op, learning_rate = get_model(
            feed_previous=True)
        saver.restore(sess, './model/EvilCalf')
        sys.stdout.write("> ")
        sys.stdout.flush()
        input_seq = sys.stdin.readline()
        while input_seq:
            input_seq = input_seq.strip()
            input_id_list = get_id_list_from(input_seq)
            if len(input_id_list):
                sample_encoder_inputs, sample_decoder_inputs, sample_target_weights = seq_to_encoder(
                    ' '.join([str(v) for v in input_id_list]))

                input_feed = {}
                for l in range(input_seq_len):
                    input_feed[encoder_inputs[l].name] = sample_encoder_inputs[l]
                for l in range(output_seq_len):
                    input_feed[decoder_inputs[l].name] = sample_decoder_inputs[l]
                    input_feed[target_weights[l].name] = sample_target_weights[l]
                input_feed[decoder_inputs[output_seq_len].name] = np.zeros([2], dtype=np.int32)

                outputs_seq = sess.run(outputs, input_feed)
                outputs_seq = [int(np.argmax(logit[0], axis=0)) for logit in outputs_seq]
                if EOS_ID in outputs_seq:
                    outputs_seq = outputs_seq[:outputs_seq.index(EOS_ID)]
                outputs_seq = [wordToken.id2word(v) for v in outputs_seq]
                print(" ".join(outputs_seq))
            else:
                print("换句话吧,这句话无解,GG")

            sys.stdout.write("> ")
            sys.stdout.flush()
            input_seq = sys.stdin.readline()


if __name__ == "__main__":
    predict()

\end{lstlisting}
\end{appendices}