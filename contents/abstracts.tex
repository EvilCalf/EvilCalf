\begin{chineseabstract}
    {对话模型;聊天机器人;序列到序列;深度学习;机器学习}面向任务的聊天机器人可以用于自动完成特定的任务,例如完成自动应答的闲聊功能。而实现这样的功能可能会很困难,需要大量的深度学习的知识以及模型规则的设计。这篇论文的重点是设计评价基于神经网络的模型来创建端到端的中文聊天机器人,该聊天机器人可以实现针对社交软件的应答回复的自动化服务。为此,实现并训练了一个序列到序列模型。当前聊天机器人模型虽然并不能作为完美的独立系统来使用,但是本文会针对性的将其与社交软件相结合,做到一定程度的优化人机交互方式。从而挖掘出聊天机器人未来的发展方向以及工作的重心,以求达到完美的人机交互方式。
\end{chineseabstract}
\begin{englishabstract}
    {conversation model;chatbots;sequence to sequence;
    deep learning;machine learning}Task-oriented chatbots can be used to automate specific tasks, such as small talk to complete automatic responses. However, it may be difficult to realize such a function, which requires a lot of deep learning knowledge and the design of model rules. This paper focuses on the design and evaluation of a neural network-based model to create an end-to-end Chinese chatbot that can automate responses to social software.To this end, a sequence-to-sequence model is implemented and trained.Although the current chatbot model cannot be used as a perfect independent system, this paper will combine it with social software to optimize human-computer interaction to a certain extent. Therefore, this paper excavates the future development direction and focus of work of chatbots in order to achieve perfect human-computer interaction.
\end{englishabstract}