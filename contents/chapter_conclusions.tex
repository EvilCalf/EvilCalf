\begin{cuzchapter}{总结与未来工作}{chap:conclusions}
\section{总结}\label{sec:background}
使用序列到序列模型实现面向任务的对话模型,在日常闲聊的语料库上进行了端到端的培训,并与几个基线进行了比较。我们的模型可以在没有任何先前的领域知识的情况下学习语言以进行基本对话。

该模型最明显的问题是,当需要完全匹配使用正确的实体会遇到麻烦。该模型倾向于使用在训练期间看到的回复而不是使用来自上下文的实体。从未使用过训练期间未见过的单词,这意味着系统无法在没有再培训的情况下处理新实体。系统在这些关键任务上的失败会使其不适合用作独立的聊天机器人。然而,这些测试显示了在面向任务的对话环境中序列到序列模型的一些优点和局限性。这些知识可用于未来的测试和开发。

本论文的目的是学习以及实践研发聊天机器人,并对对话系统的各种模型进行测试分析。测试已实施系统的性能,其拥有很多优势,但也存在不容忽视的重大缺陷。

总之,我们目前的模型不能用作独立系统来成功进行完整对话,不能达到真正的人工智能。特别是中文的语言太博大精深,如果让机器学习并且理解,可能还需要更多的对分词标注技术进行提高。本论文对神经网络所训练出的模型进行一定的评估,并能显示出其在未来工作中可能发挥最大作用的地方,同时可以为未来的工作提供经验。
\section{未来工作}\label{sec:background}
本论文的工作仅限于一个相当小的领域。可能需要对其他可能更广泛的领域进行测试,以了解结果如何扩展。与本论文中讨论的模型相比,尝试其他学习模型并评估其性能也是有意义的。存在许多不同的基于神经网络的会话模型,每个会话模型都有自己的优点和缺点。

对序列到序列模型有一些可能的修改,这些修改对于聊天机器人的研发是有改进意义的。一种修改是尝试使用完全匹配,另一种修改是用更高级的算法(例如波束搜索)替换响应生成中的贪婪搜索。

本文的模型采取逐字生成响应。一种可能的改进是实现基于检索的响应生成。基于检索的推理功能应该能够提高准确性并允许更多地控制答案。因此,我们可以避免我们现在拥有的无意义和语法错误的句子。缺点是对可能的答案的限制以及创建答案集所需的额外努力,还存在增加的计算成本。

基于规则的模型在更简单的结构化任务上优于基于网络的模型,而基于网络的模型在更加非结构化的任务上表现更好。因此,制作结合两种方法的优势的混合模型将更有前途的。在某些情况下,我们真正需要的是匹配某个单词,但是如果我们从一开始就根据我们的领域知识制定规则,那么它将不存在端到端的可训练性和系统的可移植性。然而,这可能是当今实施聊天机器人系统的合理方法,因为当前的神经网络远不够准确,可能无法自行工作。

在聊天机器人的应用中,不仅仅是准确性很重要,用户在使用系统时的感受同样重要。由于本文的目的是评估使用神经网络创建聊天机器人的能力,因此应用程序几乎仅使用序列到序列模型而不是其他任何东西。因此,有许多改进可以带来增强的用户体验。一个简单的改进是计算在对话的每个步骤中我们的响应的概率,并且如果概率低于某个阈值则给用户以一种万金油的方式。必须适当设置此阈值,以使消息不会过于频繁地显示并产生额外的烦恼。有时聊天机器人的响应方式不会朝着用户想要的方向发展。当这个错误的响应进入上下文时,它可能会在进行未来预测时混淆聊天机器人。解决这个问题的方法可能是检测我们何时发生错误而不将该部分附加到上下文。

最后是因为中文的特点,本文的训练难度和英文比起来相差非常之大,同等文本下,中文的语料库训练出的维度会多很多,并且词向量之间的距离和关系的处理难度也更大。目前的NLP水平可能还不足以完美地对中文语言进行切词。也许我们需要更多的努力对中文NLP进行研究学习。

上面我们已经提到了一些基于本论文发现的想法,但是在这个研究领域的中,未来的工作还有更多的可能性等着我们去研究。
\end{cuzchapter}