\begin{cuzchapter}{绪论}{chap:introduction}

\section{简介}\label{sec:background}

聊天机器人有各种各样的可能的用途,对行业和研究人员来说都是如此。越来越多的公司也考虑在他们的消息服务中添加机器人。聊天机器人十分有用,因为他们可以随时随地地帮助客户解答问题。那么关键的问题是当前聊天机器人的的技术是否已经变得足够先进,让我们可以设计出有效的聊天机器人。

智能程度不同的聊天机器人已经存在了很多年,第一次出现在60年代,\cite{Weizenbaum:1966:ECP:365153.365168}使用简单的关键字匹配来生成答案。从那以后经过很多技术的发展,更多的聊天机器人被开发。其中它们使用着不同的技术,例如模式匹配、自然语言处理、关键字提取、机器学习和深度学习等技术。

聊天机器人研发与制作,一般来说有着两种不同的类型,分别是基于意图的聊天机器人和面向任务的聊天机器人。一般聊天机器人就是是通过模拟对话来娱乐用户,通常没有特定的最终目标。另一方面,面向任务的机器人则负责解释和执行用户的请求,并且越快越好。聊天机器人有许多可能的应用领域,尤其是面向任务的聊天机器人在实际应用中有着明确的用途,是为了帮助用户实现某些特定的目标而构建的,比如咨询服务,闲聊,心理疏导,预定机票等等。使用机器人代替人工助理可以减少特定任务所需的时间和精力。

创建聊天机器人有不同的方法。本论文将专注于基于TensorFlow库的中文聊天机器人研发,这意味着整体模型可以直接在对话数据集既语料库上进行训练,而不需要任何先前的领域知识。这种方法也不需要特性工程,因此可以跨域使用。
\section{目标}\label{sec:background}

本项目目的是研究聊天机器人设计的现状技术和评估在面向任务环境中使用基于Tensorflow深度学习的端到端的模型训练成果的系统的可能性。为此,开发出一种序列到序列的聊天机器人。

聊天机器人必须直接从语料中学习并且完成任务,无需依赖任何人工的规则设置。对聊天机器人的闲聊应答任务进行评估,得出优势和系统的局限性。在本项目中,用于训练和测试数据来自于社交软件的中的聊天记录,公开语料库以及影视字幕中的对话。最终实现聊天机器人通过hook QQ这一社交软件帮助用户完成日常闲聊,进行快速的相应答复,以完成一般的日常对话业务。

\end{cuzchapter}